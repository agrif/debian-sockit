\documentclass{sockitguide}

\begin{document}

\sockittitle{Creating a Debian SD Card for Arrow SoCKit}

This guide will walk you through the process of creating a MicroSD card
suitable for booting a SoCKit board into Debian, from scratch (insofar
as that is possible).

If you just want to get started, it is probably easier to get a pre-made
Debian image from someone. This guide just notes how that image is
created, if none already exist.

This guide is loosely based off \fnurl{\textit{Exploring the Arrow
    SoCKit Part
    II}}{https://zhehaomao.com/blog/fpga/2013/12/24/sockit-2.html}, by
Howard Mao. The rest of his tutorial is amazing, and worth a look.

\section{Preparing a Workspace}

This guide is written for use on an existing Debian Jessie installation,
with access to a MicroSD card reader. It might work in other situations
with some changes. I recommend using a virtual machine.

It would be a good idea to run this whole guide inside a new, empty
directory, just for a clean work space.

You will need about \SI{1.8}{\gibi\byte} of free space to work in. In
addition, your SD card needs to be \SI{1}{\gibi\byte} or more. If you
are working on Debian, the tools you need to follow this guide will
take up about \SI{3}{\gibi\byte} more.

To complete the build, you'll need some tools installed:
\begin{itemize}
\item \texttt{curl}, to fetch a codesigning key
\item \texttt{debootstrap}, to install the Debian root
\item \texttt{dosfstools}, to make a FAT filesystem
\item \texttt{git}, to clone a copy of the Linux kernel sources
\item \texttt{kernel-package}, to build Debian kernel packages
\item \texttt{ncurses-dev}, to build the kernel's configuration menu
\item \texttt{qemu-user-static}, for \texttt{qemu-debootstrap}, to install the Debian root
\end{itemize}

On Debian, you can install all these with:
\begin{minted}{console}
$ sudo apt-get install curl debootstrap \
    dosfstools git kernel-package ncurses-dev \
    qemu-user-static
\end{minted}

Additionally, you'll need a \fnurl{cross-compilation toolchain for
  ARM}{http://emdebian.org/tools/debian/}. To install one, create the
file \texttt{/etc/apt/sources.list.d/crosstools.list} and add the
following:
\begin{minted}[breaklines]{sources.list}
deb http://emdebian.org/tools/debian/ jessie main
\end{minted}

You will also need to install the codesigning key they use:
\begin{minted}{console}
$ curl http://emdebian.org/tools/debian/\
  emdebian-toolchain-archive.key \
  | sudo apt-key add -
\end{minted}

Now, you can add \texttt{armhf} to your architecture list, update your package list, and install the toolchain:
\begin{minted}{console}
$ sudo dpkg --add-architecture armhf
$ sudo apt-get update
$ sudo apt-get install crossbuild-essential-armhf
\end{minted}

\section{Preparing the SD Card}

First, you will need to identify which device in \texttt{/dev}
represents your SD card. One way to do this is to insert a new SD
card, wait for your desktop environment to automatically mount it,
then run the \texttt{mount} command to see what's new.

Be sure you get this right! If you accidentally try to run the following
commands on the wrong device, you could destroy data on your hard
drives.

If you don't have an SD card handy, or don't want to try this, you can
also create a disk image and set it up as a loopback device:
\begin{minted}{console}
$ dd if=/dev/zero of=my_debian_image.img \
  bs=1M count=1024
$ sudo losetup --show -fP my_debian_image.img
\end{minted}

The last command will print a line (probably \texttt{/dev/loop0});
this is your device.

From now on, I will use \texttt{/dev/sdcard} as my device
name. Subtitute your actual device name as needed.

\subsection{Partitioning}

The SoCKit needs at least three partitions on the SD card to boot. These
are:
\begin{itemize}
\item a bootloader, \SI{1}{\mebi\byte}, starting at sector 2048, with ID 0xA2
\item a boot information partition, FAT formatted, as partition \#1
\item a root partition, as partition \#2
\end{itemize}

Since partitions \#1 and \#2 are taken by boot and root, we will stick
the bootloader partition in slot \#3. We'll also stick the root
partition at the end of the card, to make it easier to resize later
(if we were making a disk image).

Let's launch \texttt{fdisk} and make these partitions. First off, we
need to clear the existing table (if any):
\begin{minted}{console}
$ sudo fdisk /dev/sdcard
Welcome to fdisk (util-linux 2.25.2).
...

Command: o
\end{minted}

Now, we need to make the bootloader partition(\#3):
\begin{minted}[breaklines]{console}
Command: n
Select: p
Partition number: 3
First sector: 2048
Last sector, +sectors or +size: +2048

Created a new partition 3 of type 'Linux' and of size 1 MiB.

Command: t
Hex code: a2
Changed type of partition 'Linux' to 'unknown'.
\end{minted}

Next up is the boot information partition (\#1):
\begin{minted}[breaklines]{console}
Command: n
Select: p
Partition number: 1
First sector: <enter>
Last sector, +sectors or +size: +128M

Created a new partition 1 of type 'Linux' and of size 128MiB.

Command: t
Partition number: 1
Hex code: 0b
Changed type of partition 'Linux' to 'W95 FAT32'.
\end{minted}

And finally the root partition (\#2):
\begin{minted}[breaklines]{console}
Command: n
Select: p
Partition number: 2
First sector: <enter>
Last sector, +sectors or +size: <enter>

Created a new partition 2 of type 'Linux' and of size <...>.
\end{minted}

In principle, you can do anything you want with the fourth (unused)
partition, and it won't affect the boot. We choose to ignore it, and use
all remaining free space for the root partition (\#2).

Before we're done, you should print out the table you've constructed and
inspect it, then write it out to disk.
\begin{minted}[breaklines]{console}
Command: p
<inspect table, then...>  
Command: w
Command: q
\end{minted}

Once \texttt{fdisk} exits, you should have three shiny new partitions
to format.

\subsection{Formatting}

Our boot information partition (\#1) needs a FAT filesystem:
\begin{minted}{console}
$ sudo mkfs.vfat -n "SOCKIT BOOT" /dev/sdcard1
\end{minted}

For our root partition (\#2), we will use Ext4:
\begin{minted}{console}
$ sudo mkfs.ext4 -L "Debian Jessie" \
  /dev/sdcard2
\end{minted}

Our bootloader partition (\#3) isn't a filesystem, it's a binary blob
containing \fnurl{U-Boot}{http://www.denx.de/wiki/U-Boot}, and a bit
of code to interface it with the SoCKit board. At the moment, I have
\fnurl{no
  idea}{https://rocketboards.org/foswiki/view/Documentation/GSRD151SDCardArrowSoCKitEdition}
how this blob is created, but it should have been included alongside
this guide as \texttt{extras/uboot.img}. We will write that to the
disk now:
\begin{minted}{console}
$ sudo dd if=extras/uboot.img of=/dev/sdcard3 \
  bs=4k
\end{minted}

Now that we have the SD card set up, we can start putting data on it.

\section{Building a Kernel}

Our boot information filesystem needs to contain at least two files:
\begin{itemize}
\item \texttt{zImage}, a compressed Linux kernel
\item \texttt{socfpga.dtb}, a \fnurl{device tree}{http://elinux.org/Device_Tree} describing board layout
\end{itemize}

Both of these can be created from the Linux kernel source tree, which
we will clone from Altera via \texttt{git}. We'll use a shallow clone
and only check out the specific tag we need, as the full Linux git
repository is gigantic.

\begin{minted}{console}
$ git clone --depth=1 -b socfpga-4.10 \
  https://github.com/altera-opensource/\
  linux-socfpga.git
\end{minted}

This tag is known to work, but may be old. It might be worth trying newer
tags.

Before we continue, we'll set up a quick \texttt{bash} alias to make
cross-building for ARM easier:
\begin{minted}{console}
$ alias armmake="make \
  CROSS_COMPILE=arm-linux-gnueabihf- ARCH=arm"
\end{minted}

Ok. Move into the shiny new kernel sources, and set up the default
configuration for the SoCKit:
\begin{minted}{console}
$ cd linux-socfpga
$ armmake socfpga_defconfig
\end{minted}

This is a good start, but there are a few kernel options we need to
enable to get Debian Jessie running:
\begin{minted}{console}
$ armmake menuconfig
\end{minted}

You should now have a configuration menu. Navigate to, and enable, the
following options. Now would be a great time to enable any other kernel
stuff you want.

\begin{minted}[breaklines]{text}
General setup  --->
    [*] open by fhandle syscalls
Enable the block layer  --->
    [*] Support for large (2TB+) block devices and files
    [*] Block layer SG support v4
File systems  --->
    [*]     Ext2 Security Labels
    [*]     Ext3 Security Labels
    [*]   Ext4 Security Labels
    [*] Inotify support for userspace
\end{minted}

Exit the menu, and be sure to save! Now we are going to use
\fnurl{\texttt{make-kpkg}}{http://man.he.net/man1/make-kpkg} to build
a Debian kernel package from these sources, and along the way, our
\texttt{zImage}. This can take a \textit{long time}, so we will build
in parallel: change the 2 in \texttt{-j2} to the number of
CPUs on your build machine.

\begin{minted}{console}
$ fakeroot make-kpkg --arch armhf \
  --cross-compile arm-linux-gnueabihf- \
  -j2 --revision 1 binary-indep binary-arch
\end{minted}

This stuck a bunch of Debian packages in the directory above your kernel
source. We'll use those later. Now we just need the device trees:
\begin{minted}{console}
$ armmake dtbs
\end{minted}

Move back outside the kernel source, and mount our boot information
partition (\#1) somewhere temporary:
\begin{minted}{console}
$ cd ..
$ mkdir mnt
$ sudo mount /dev/sdcard1 mnt/
\end{minted}

Now we copy the two files we need into place:
\begin{minted}{console}
$ sudo cp linux-socfpga/arch/arm/boot/zImage \
  mnt/zImage
$ sudo cp linux-socfpga/arch/arm/boot/dts/\
  socfpga_cyclone5_sockit.dtb mnt/socfpga.dtb
\end{minted}

There are a few other things you can add to this partition, in
particular a
\fnurl{\texttt{u-boot.scr}}{https://lists.rocketboards.org/pipermail/rfi/2014-April/001602.html}
file, which U-Boot reads and executes at boot. This can be used to
customize the boot process, but it's not required, so we skip it
here. An example is provided alongside this guide in \texttt{extras/}.

Also, since you may be modifying both \texttt{zImage} and
\texttt{socfpga.dtb} often in the future, you might want to make a
\texttt{backup/} subdirectory and keep copies of both in there, so you
always have a set that you know works.

Unmount the boot information partition:
\begin{minted}{console}
$ sudo umount mnt/
\end{minted}

Your boot information partition is now finished.

Please note that cross-compiling a kernel like this works well enough
to get the board to boot, but ultimately is a little bit broken. Once
you have the board working at the end of this guide, you will want to
run through this section again \textit{on the board itself}, to
generate a truly native kernel build.

\section{Installing Debian}

We will be using the \texttt{qemu-debootstrap} tool to install Debian on to the
root partition. First, though, we have to mount the root partition
(\#2):
\begin{minted}{console}
$ sudo mount /dev/sdcard2 mnt/
\end{minted}

We'll be installing Debian Jessie from the official mirror, but other
mirrors and releases will likely work:
\begin{minted}{console}
$ sudo qemu-debootstrap --arch=armhf jessie \
  mnt/ http://ftp.debian.org/debian/
\end{minted}

Lastly, we'll install the Debian kernel packages we made earlier. All we
need right now are the kernel modules and headers, but we copy the
others into \texttt{/root} in case we need them later.

\begin{minted}{console}
$ sudo cp *.deb mnt/root/
$ sudo chroot mnt/ dpkg -i \
  /root/linux-headers-*.deb
$ sudo chroot mnt/ dpkg -i \
  /root/linux-image-*.deb
\end{minted}

As of now, you have a fully functioning Debian system, but there are a
handful of niceties to finish up.

\subsection{Setting Up Users}

Running things as \texttt{root} all the time is (generally speaking) a bad
idea. To create a new user for everyday work:
\begin{minted}{console}
$ sudo chroot mnt/ adduser --add_extra_groups \
  <username>
\end{minted}

To give this user access to \texttt{sudo} so they can become root:
\begin{minted}{console}
$ sudo chroot mnt/ adduser <username> sudo
\end{minted}

Since we have this user, we'll lock root logins:
\begin{minted}{console}
$ sudo chroot mnt/ passwd -l root
\end{minted}

\subsection{Installing Tools}

There are a couple of Debian packages you would probably like to have:
\begin{itemize}
\item \texttt{locales}, mostly to get rid of errors, but also to handle locales
\item \texttt{ntp}, to set the correct time on boot via the Internet
\item \texttt{openssh-server}, for remote shell access
\item \texttt{sudo}, to get root access
\end{itemize}

To install these:
\begin{minted}{console}
$ sudo chroot mnt/ apt-get install locales ntp \
  openssh-server sudo
\end{minted}

These commands have started a few services accidentally. You can kill
them with:
\begin{minted}{console}
$ sudo killall mnt/usr/bin/qemu-arm-static
\end{minted}

By default, the SSH server allows logging in remotely as root. This is a
pretty bad idea, so edit \texttt{mnt/etc/ssh/sshd\_config} and change the
following line:
\begin{minted}{apacheconf}
PermitRootLogin without-password
# change to:
PermitRootLogin no
\end{minted}

If you care to, you can change the default timezone from UTC to
something more appropriate:
\begin{minted}{console}
$ sudo chroot mnt/ dpkg-reconfigure tzdata
\end{minted}

Finally, reconfigure locales to use the \texttt{en\_US.UTF-8} locale:
\begin{minted}{console}
$ sudo chroot mnt/ dpkg-reconfigure locales
\end{minted}

\subsection{Setting Up Networking}

To have the ethernet device automatically set itself up with DHCP on
boot (which is usually what you want), edit the file
\texttt{mnt/etc/network/interfaces} and add the following lines:
\begin{minted}{text}
allow-hotplug eth0
iface eth0 inet dhcp
\end{minted}

By default, the new Debian install will have the same hostname as the
host you built it on. This can be confusing. Edit
\texttt{mnt/etc/hostname} to change it, then edit
\texttt{mnt/etc/hosts} to add the new hostname:
\begin{minted}[breaklines]{text}
127.0.0.1       localhost <hostname>
::1             localhost ip6-localhost ip6-loopback <hostname>
ff02::1         ip6-allnodes
ff02::2         ip6-allrouters
\end{minted}

\subsection{Fix APT Sources}

\texttt{debootstrap} installs a very minimal list of sources for APT
to use. To fix it, open \texttt{mnt/etc/apt/sources.list} and replace
the whole thing with:
\begin{minted}[breaklines]{sources.list}
deb http://ftp.us.debian.org/debian jessie main contrib non-free
deb-src http://ftp.us.debian.org/debian jessie main contrib non-free

deb http://security.debian.org/ jessie/updates main contrib non-free
deb-src http://security.debian.org/ jessie/updates main contrib non-free

# jessie-updates, previously known as 'volatile'
deb http://ftp.us.debian.org/debian jessie-updates main contrib non-free
deb-src http://ftp.us.debian.org/debian jessie-updates main contrib non-free
\end{minted}

Make sure these new mirrors work by updating your package list:
\begin{minted}{console}
$ sudo chroot mnt/ apt-get update
\end{minted}

\section{Finishing Up}

Now that everything's in place, we can unmount the root partition:
\begin{minted}{console}
$ sudo umount mnt/
\end{minted}

If you are writing to a disk image over a loopback device, now is the
time to destroy that, too:
\begin{minted}{console}
$ sudo losetup -d /dev/sdcard
\end{minted}

Your SD card is finished! Go and boot it, and watch over the USB serial
console. When the time comes, you will be able to log in with username
and password you set earlier.

If you created the card using an image, now would be an excellent time
to expand the root partition to cover the rest of the card, and resize
the root filesystem.

\end{document}
